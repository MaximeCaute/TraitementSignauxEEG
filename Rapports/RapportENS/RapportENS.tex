\documentclass[a4paper]{article}
\usepackage[utf8]{inputenc}
\usepackage[francais]{babel}

\usepackage{amsmath}


\newtheorem{myDef}{Définition}
\newtheorem{ex}{Exemple}

%\usepackage{biblatex}

%\addbibresource{biblio.bib}



\begin{document}

\title{Transport optimal et apprentissage automatique pour débruitage de signaux EEG \\
			Rapport de projet de M1}
\author{Maxime Cauté}
%\affil{ENS Rennes} %IRISA & EMPENN ?


\maketitle

\subsection{Environnement}

Les données ont été manipulées à l'aide de la boîte à outils MNE. %TODO Citer refs : https://mne.tools/0.11/cite.html

\subsection{Transport Optimal (discret)}

Le transport optimal est une méthode d'optimisation.

Formellement, pour deux distributions de probabilité (deux vecteurs dans $[0,1]$ dont la somme vaut 1) $v, v'$, cela revient à calculer un élément minimal de l'ensemble $\theta$ suivant :

	$$ \Theta = \{ T \in M_m(R), \sum_{j=1}^m t_{i,j} = v_i \wedge \sum_{i=1}^m t_{i,j} = v'_j\}$$

On cherche ensuite à minimiser :

	$$ T_0 = argmin_{T \in \Theta} <T, C>_F = argmin_{T \in \Theta} \sum_{i,j} t_{i,j}c_{i,j} $$

La matrice $C$ est une fonction de coût, dont les composantes $c_{i,j}$ représentent le coût de transport de $v_i$ vers $v'_j$. Si $C$ est symétrique, $<T_0,C>$ devient la distance du terrassier entre $v$ et $v'$.




\section*{Remerciements}

Pierre Maurel, Julie Coloigner, Giulia Lioi de l'équipe Empenn pour l'encadrement


\end{document}
